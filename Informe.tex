%%%%%%%%%%%%%%%%%%%%%%%%%%%%% Define Article %%%%%%%%%%%%%%%%%%%%%%%%%%%%%%%%%%
\documentclass{article}
%%%%%%%%%%%%%%%%%%%%%%%%%%%%%%%%%%%%%%%%%%%%%%%%%%%%%%%%%%%%%%%%%%%%%%%%%%%%%%%

%%%%%%%%%%%%%%%%%%%%%%%%%%%%% Using Packages %%%%%%%%%%%%%%%%%%%%%%%%%%%%%%%%%%
\usepackage{geometry}
\usepackage{graphicx}
\usepackage{amssymb}
\usepackage{amsmath}
\usepackage{amsthm}
\usepackage{empheq}
\usepackage{mdframed}
\usepackage{booktabs}
\usepackage{lipsum}
\usepackage{graphicx}
\usepackage{color}
\usepackage{psfrag}
\usepackage{pgfplots}
\usepackage{bm}
\usepackage{fancyhdr}
\usepackage{fontspec}
\usepackage[spanish]{babel}
\usepackage{datetime}
\usepackage{enumitem}
%%%%%%%%%%%%%%%%%%%%%%%%%%%%%%%%%%%%%%%%%%%%%%%%%%%%%%%%%%%%%%%%%%%%%%%%%%%%%%%

\setmainfont{calibri}[BoldFont = Calibri Bold]

%%%%%%%%%%%%%%%%%%%%%%%%%% Page Setting %%%%%%%%%%%%%%%%%%%%%%%%%%%%%%%%%%%%%%%
\geometry{a4paper}
\geometry{top=1.9cm, bottom=3.67cm, left=1.9cm, right=1.32cm}
\pagestyle{fancy}
\fancyhf{}
\lhead{\textbf{UNCuyo – Ing. Mecatrónica}\\Mendoza - Argentina}
\chead{\textbf{311 – AUTOMÁTICA Y MÁQUINAS ELÉCTRICAS}\\ 
\textbf{PROYECTO GLOBAL INTEGRADOR}}
\rhead{Alumnos: Mamani , Vignolo\\  \today}
\rfoot{Página \thepage{} de \pageref{LastPage}}
\renewcommand{\footrulewidth}{0.4pt}
\renewcommand{\headrulewidth}{0.4pt}
%%%%%%%%%%%%%%%%%%%%%%%%%%%%%%%%%%%%%%%%%%%%%%%%%%%%%%%%%%%%%%%%%%%%%%%%%%%%%%%

\begin{document}

\begin{titlepage}
    \centering
    \vspace*{3cm}
    \Huge\textbf{\textsc{Control de Accionamiento de CA con Motor Sincronico de Imanes Permanentes}}\\
    \vspace{1.5cm}
    \large{Subtitulo}\\
    \vspace{10cm}
    \large{Autors:}\\
    \large{\textbf{Alan Vignolo\\Brandon Mamani}}\\
    \vfill
    \the\year\\ % o puedes especificar una fecha específica en lugar de \today
\end{titlepage}

\part*{Introducción}

\lipsum

\newpage
\part*{Desarrollo}

\section*{Modelado, Análisis y Simulación dinámica del SISTEMA FÍSICO a Lazo Abierto}
\subsection*{Modelo matemático equivalente del subsistema mecánico completo}

Subsistema mecánico del motor de CA trifásica síncrono, con el rotor referido al estator - sistema inercial de referencia:
\begin{equation}
    \label{eq:1.1}
    J_{m}\dot{\omega}_{m}(t) = T_{m}(t)-b_{m}\omega_{m}(t)-T_{d}(t) \\
\end{equation}
\begin{equation}
    \label{eq:1.2}
    \dot{\theta}_{m} = \omega_{m}
\end{equation}

Subsistema de tren de transmisión:
\begin{equation}
    \label{eq:2.1}
    \omega_{l}(t) = \frac{1}{r}\omega_{m}(t) \\
\end{equation}
\begin{equation}
    \label{eq:2.2}
    T_{q}(t) = rT_{d}(t) \\
\end{equation}

Subsistema de la carga mecánica:
\begin{equation}
    \label{eq:3.1}
    J_{l}\dot{\omega}_{l}(t) = T_{q}(t)-b_{l}\omega_{l}(t)-T_{l}(t) \\
\end{equation}
\begin{equation}
    \label{eq:3.2}
    \dot{\theta}_{l} = \omega_{l}\\
\end{equation}

Sistema mecánico equivalente completo:
\begin{equation}
    \label{eq:4.1}
    (J_{m}+\frac{J_{l}}{r^2})\dot{\omega}_{m}(t) = T_{m}(t)-(b_{m}+\frac{b_{l}}{r^2})\omega_{m}(t)-\frac{T_{d}(t)}{r} \\
\end{equation}

\begin{equation}
    J_{eq}\dot{\omega}_{m}(t) = T_{m}(t)-b_{eq}\omega_{m}(t)-\frac{T_{d}(t)}{r} \\
\end{equation}

\begin{equation}
    \dot{\omega}_{m}(t) = \frac{1}{J_{eq}}[T_{m}(t) - b_{eq}\omega_{m}(t) - \frac{T_{l}(t)}{r}]\\
\end{equation}

\begin{equation}
\begin{cases}
    \begin{bmatrix}
        \dot{\theta}_{m}(t)\\
        \dot{\omega}_{m}(t)\\
    \end{bmatrix}
    =
    \begin{bmatrix}
        0 & 1\\
        0 & -\frac{b_{eq}}{J_{eq}}\\
    \end{bmatrix}
    \begin{bmatrix}
        \theta_{m}(t)\\
        \omega_{m}(t)\\
    \end{bmatrix}
    +
    \begin{bmatrix}
        0 & 0\\
        \frac{1}{J_{eq}} & -\frac{1}{J_{eq}r}\\
    \end{bmatrix}
    \begin{bmatrix}
        T_{m}(t)\\
        T_{l}(t)\\
    \end{bmatrix}\\
    y(t) = 
    \begin{bmatrix} 1 & 0\\ \end{bmatrix} 
    \begin{bmatrix} \theta_{m}(t)\\ \omega_{m}(t)\\ \end{bmatrix}
\end{cases}
\end{equation}

Con este modelo matemático equivalente referido al eje del motor tiene como 
ventaja que no presenta backlash, ademas no hay que considerar el efecto de 
la elasticidad torsional de la transmisión.

\subsection*{Modelo dinámico del sistema físico completo}
\subsubsection*{Modelo global no lineal (NL)}

El modelo global no lineal considera tanto el sistema mecánico, previamente desarrollado
, como los subsistemas electromagnético y térmico.\\

En primer lugar, nos enfocaremos en el subsistema electromagnético, teniendo en cuenta que 
se utiliza un motor síncrono de corriente alterna (CA) trifásico con excitación de imanes 
permanentes. El estator esta conectado en estrella con bornes abc accesible y neutro no accesible.
Consideramos que la carga de cada fase sera equivalente de forma que la conexión estrella este equilibrada.

Ecuaciones de tensión en coordenadas abc:
\begin{equation}
    \begin{aligned}
        V_{a_{s}}(t) &= R_{s}i_{a_{s}}(t) + \frac{d\lambda_{a_{s}}}{dt}\\
        V_{b_{s}}(t) &= R_{s}i_{b_{s}}(t) + \frac{d\lambda_{b_{s}}}{dt}\\
        V_{c_{s}}(t) &= R_{s}i_{c_{s}}(t) + \frac{d\lambda_{c_{s}}}{dt}\\
    \end{aligned}
\end{equation}

Mediante la transformación de Park, que consiste en premultiplicar por la matriz de Park se obtiene:
\begin{equation}
    \begin{aligned}
        V_{q_{s}}(t) &= R_{s}i_{q_{s}}(t) + L_{q}\dot{i}_{q_{s}}^r(t) + [\lambda_{m}^r + L_{d}i_{d_{s}}(t)]\omega_{r}(t)\\
        V_{d_{s}}(t) &= R_{s}i_{d_{s}}(t) + L_{d}\dot{i}_{d_{s}}^r(t)  - L_{q}i_{q_{s}}(t)\omega_{r}(t)\\
        V_{0_{s}}(t) &= R_{s}i_{0_{s}}(t) + L_{ls}\dot{i}_{0_{s}}^r(t) \\
    \end{aligned}
\end{equation}

\begin{equation}
    \begin{cases}
        \dot{i}_{q_{s}}(t) = \frac{1}{L_{q}}[V_{q_{s}}^r(t) - R_{s}i_{q_{s}}^r(t) - P_{p}\omega_{m}(t)[L_{d}i_{d_{s}}^r(t)+\lambda_{m}^r]]  \\
        \dot{i}_{d_{s}}(t) = \frac{1}{L_{d}}[V_{d_{s}}^r(t) - R_{s}i_{d_{s}}^r(t) + P_{p}\omega_{m}(t)L_{q}i_{q_{s}}^r(t)]  \\
        \dot{i}_{0_{s}}(t) = \frac{1}{L_{ls}}[V_{0_{s}}^r(t) - R_{s}i_{0_{s}}^r(t)]  \\
    \end{cases}
\end{equation}

%Dudas:
Dada la conexión que presenta el motor, podemos suponer que la corriente $i_{0_{s}}$ es nula

El subsistema térmico
Solo se consideran las pérdidas eléctricas resistivas causadas por el efecto Joule (calor),
 despreciando las pérdidas magnéticas en el núcleo y las transferencia de calor por conducción 
y convección natural. La potencia de pérdidas calóricas está dada por:\\
\begin{equation}
    P_{s_{perd}}(t) =  \frac{3}{2}R_{s}(t)(i_{q_{s}}^r(t)^2+i_{d_{s}}^r(t)^2+2.i_{0_{s}}(t))\\
\end{equation}
\par El balance térmico en el estator:
\begin{equation}
    P_{s_{perd}}(t) = C_{s}\dot{T}_{s}(t) + \frac{T_{s}(t)-T_{amb}(t)}{R_{ts-amb}}
\end{equation}

Torque electromagnético:
\begin{equation}
    T_{m}(t) = \frac{3}{2}P_{p}[\lambda_{m}^r+i_{d_{s}}^r(t)(L_{d}-L_{q})]i_{q_{s}}^r(t)
\end{equation}

El modelo global:
\begin{equation}
    \begin{cases}
        \dot{\theta}_{m}(t) = \omega_{m}(t)\\
        \dot{\omega}_{m}(t) = \frac{1}{J_{eq}}[\frac{3}{2}P_{p}[\lambda_{m}^r+i_{d_{s}}^r(t)(L_{d}-L_{q})]i_{q_{s}}^r(t) - b_{eq}\omega_{m}(t) - \frac{T_{l}(t)}{r}]\\
        \dot{i}_{q_{s}}(t) = \frac{1}{L_{q}}[V_{q_{s}}^r(t) - R_{s}i_{q_{s}}^r(t) - P_{p}\omega_{m}(t)[L_{d}i_{d_{s}}^r(t)+\lambda_{m}^r]]  \\
        \dot{i}_{d_{s}}(t) = \frac{1}{L_{d}}[V_{d_{s}}^r(t) - R_{s}i_{d_{s}}^r(t) + P_{p}\omega_{m}(t)L_{q}i_{q_{s}}^r(t)]  \\
        \dot{T}_{s}(t) = \frac{1}{C_{ts}}[\frac{3}{2}r_{s}({i_{a_{s}}^r(t)}^2+{i_{a_{s}}^r(t)}^2+2i_{0_{s}}(t)^2)-\frac{T_{s}(t)-T_{amb}(t)}{R_{ts-amb}}]\\
    \end{cases}
\end{equation}


%-------------------------- 2B------------------------------------------%

\subsubsection*{Modelo global linealizado con parámetros variables (LPV)}

% Poner toda la demo? Seguir corrigiendo ya que tengo inconsistencias con I0

Cuasi-estacionario
\begin{equation}
    \begin{cases}
        \dot{\theta}_{m0} = \omega_{m0} = cte\\
        \dot{\omega}_{m0} = \frac{1}{J_{eq}}[\frac{3}{2}P_{p}[\lambda_{m}^r+i_{d_{s0}}^r(L_{d}-L_{q})]i_{q_{s0}}^r - b_{eq}\omega_{m0} - \frac{T_{l0}}{r}] = 0\\
        \dot{i}_{q_{s0}} = \frac{1}{L_{q}}[V_{q_{s0}}^r - R_{s}i_{q_{s0}}^r - P_{p}\omega_{m0}[L_{d}i_{d_{s0}}^r+\lambda_{m}^r]] = 0\\
        \dot{i}_{d_{s0}} = \frac{1}{L_{d}}[V_{d_{s0}}^r - R_{s}i_{d_{s0}}^r + P_{p}\omega_{m0}L_{q}i_{q_{s0}}^r] = 0 \\
        \dot{T}_{s0} = \frac{1}{C_{ts}}[\frac{3}{2}r_{s}({i_{a_{s0}}^r}^2+{i_{a_{s0}}^r}^2+2i_{0_{s0}}^2)-\frac{T_{s0}-T_{amb0}}{R_{ts-amb}}] = 0\\
    \end{cases}
\end{equation}

Parámetros Variables
\begin{equation}
    \begin{cases}
        \Delta\dot{\theta}_{m}(t) = \Delta\omega_{m}(t)\\
        \Delta\dot{\omega}_{m}(t) = \frac{1}{J_{eq}}[\frac{3}{2}P_{p}\{[\lambda_{m}^r + i_{d_{s0}}^r(L_{d}-L_{q})]\Delta i_{q_{s}}^r(t) + \Delta i_{d_{s}}^r(t)(L_{d}-L_{q})]i_{q_{s0}}^r\}- b_{eq}\Delta\omega_{m}(t) - \frac{\Delta T_{l}(t)}{r}]\\
        \Delta\dot{i}_{q_{s}}(t) = \frac{1}{L_{q}}[\Delta V_{q_{s}}^r(t) - R_{s}\Delta i_{q_{s}}^r(t) - P_{p}\Delta\omega_{m}(t)[L_{d}i_{d_{s0}}^r+\lambda_{m}^r] - P_{p}\omega_{m0}L_{d}\Delta i_{d_{s}}^r(t)] \\
        \Delta\dot{i}_{d_{s}}(t) = \frac{1}{L_{d}}\Delta [V_{d_{s}}^r(t) - R_{s}\Delta i_{d_{s}}^r(t) + P_{p}\Delta\omega_{m}(t)L_{q}i_{q_{s0}}^r + P_{p}\omega_{m0}L_{q}\Delta i_{q_{s}}^r(t)]  \\
        \Delta\dot{T}_{s}(t) = \frac{1}{C_{ts}}\{\frac{3}{2}r_{s}[{2i_{a_{s}}^r\Delta i_{a_{s}}^r(t)} + 2i_{a_{s}}^r\Delta i_{a_{s}}^r(t)+4i_{0_{s}}\Delta i_{0_{s}}(t)]-\frac{\Delta T_{s}(t)-\Delta T_{amb}(t)}{R_{ts-amb}}\}\\
    \end{cases}
\end{equation}

\begin{multline}
    \begin{bmatrix}
        \Delta\dot{\theta}_{m}(t)\\
        \Delta\dot{\omega}_{m}(t)\\
        \Delta\dot{i}_{q_{s}}(t)\\
        \Delta\dot{i}_{d_{s}}(t)\\
        \Delta\dot{T}_{s}(t)\\
    \end{bmatrix} =
    \begin{bmatrix}
        0 & 1 & 0 & 0 & 0 & 0 \\
        0 & -b_{eq} & \frac{3}{2}P_{p}\frac{[\lambda_{m}^r + i_{q_{s0}}^r(t)(L_{d}-L_{q})]}{J_{eq}} & \frac{3}{2}\frac{P_{p}(L_{d}-L_{q})i_{q_{s}}^r}{J_{eq}} & 0 & 0\\
        0 & -\frac{P_{p}(\lambda_{m}^r + L_{d}i_{d_{s0}}^r)}{L_{q}} & -\frac{R_{s}}{L_{q}} & -\frac{L_{d}P_{p}\omega_{m0}}{L_{q}} & 0 & 0\\
        0 & \frac{P_{p}i_{q_{s0}}^r L_{q}}{L_{d}} & \frac{L_{q}P_{p}\omega_{m0}}{L_{d}} & -\frac{R_{s}}{L_{d}} & 0 & 0\\
        0 & 0 & \frac{3R_{s}}{C_{ts}}i_{q_{s0}}^r & \frac{3R_{s}}{C_{ts}}i_{d_{s0}}^r & XXXX & -\frac{1}{C_{ts}R_{ts-amb}}\\
    \end{bmatrix}
    \begin{bmatrix}
        \Delta{\theta}_{m}(t)\\
        \Delta{\omega}_{m}(t)\\
        \Delta{i}_{q_{s}}(t)\\
        \Delta{i}_{d_{s}}(t)\\
        \Delta{T}_{s}(t)\\
    \end{bmatrix}\\  + 
    \begin{bmatrix}
        0 & 0 & 0 & 0 & 0\\
        -\frac{1}{rJ_{eq}} & 0 & 0 & 0 & 0\\
        0 & \frac{1}{L_{q}} & 0 & 0 & 0\\
        0 & 0 & \frac{1}{L_{d}} & 0 & 0\\
        0 & 0 & 0 & 0 & \frac{1}{C_{ts}R_{ts-amb}}\\
    \end{bmatrix}
    \begin{bmatrix}
        \Delta{T}_{l}(t)\\
        \Delta{V}_{q_{s}}(t)\\
        \Delta{V}_{d_{s}}(t)\\
        XXXX\\
        \Delta T_{amb}(t)\\
    \end{bmatrix} 
\end{multline}


%-------------------- 2C --------------------------%

\subsubsection*{Linealización por Realimentación NL}

Se busca un modelo simplificado lineal invariante LTI equivalente al modelo NL del sistema, 
para ello se propone un controlador de realimentación de estados que permita linealizar el 
modelo global NL obtenido anteriormente, por lo que se propone:

\begin{itemize}
    \item Aplicar la estrategia de “Control Vectorial con campo orientado” la cual 
    consiste en desacoplar los canales de flujo magnético y torque, lo cual provoca
    un forzamiento de $i_{d_{s}}=0$ 
    \item Desacoplar el subsistema térmico, ya que se considera que hay variaciones despreciables 
    de Rs en el rango de temperaturas de trabajo.
\end{itemize}

\begin{enumerate}[label=\roman*.]
    \item Ecuaciones vectoriales/matriciales LTI de estado y de salida. 
    Matrices del modelo LTI equivalente.

    \begin{equation}
        \begin{cases}
            \dot{\theta}_{m}(t) = \omega_{m}(t)\\
            \dot{\omega}_{m}(t) = \frac{1}{J_{eq}}[\frac{3}{2}P_{p}\lambda_{m}^ri_{q_{s}}^r(t) - b_{eq}\omega_{m}(t) - \frac{T_{l}(t)}{r}]\\
            \dot{i}_{q_{s}}(t) = \frac{1}{L_{q}}[V_{q_{s}}^r(t) - R_{s}i_{q_{s}}^r(t) - P_{p}\omega_{m}(t)\lambda_{m}^r]  \\
        \end{cases}
    \end{equation}

    \begin{equation}
        \begin{cases}
            \begin{bmatrix}
                \dot{\theta}_{m}(t)\\
                \dot{\omega}_{m}(t)\\
                \dot{i}_{q_{s}}(t)\\
            \end{bmatrix} =
            \begin{bmatrix}
                0 & 1 & 0\\
                0 & -\frac{b_{eq}}{J_{eq}} & \frac{3}{2}\frac{P_{p}\lambda_{m}^r}{J_{eq}}\\
                0 & -\frac{P_{p}\lambda_{m}^r}{L_{q}} & -\frac{R_{s}}{L_{q}}\\
            \end{bmatrix}
            \begin{bmatrix}
                {\theta}_{m}(t)\\
                {\omega}_{m}(t)\\
                {i}_{q_{s}}(t)\\
            \end{bmatrix} +
            \begin{bmatrix}
                0\\
                0\\
                \frac{1}{L_{eq}}\\
            \end{bmatrix} V_{q_{s}}^r(t) +
            \begin{bmatrix}
                0\\
                -\frac{1}{rJ_{eq}}\\
            \end{bmatrix} T_{l}(t)\\
            y(t) =
            \begin{bmatrix}
                1 & 0 & 0\\
            \end{bmatrix}
            \begin{bmatrix}
                {\theta}_{m}(t)\\
                {\omega}_{m}(t)\\
                {i}_{q_{s}}(t)\\
            \end{bmatrix}
        \end{cases}
    \end{equation}
        
    \item Segundo ítem
    
    
    
    \item Tercer ítem



    \item Cuarto ítem
    


    \item Quinto ítem



\end{enumerate}


\newpage
\part*{Conclusiones}

\newpage
\part*{Referencias}
\lipsum{}
\label{LastPage}    
\end{document}