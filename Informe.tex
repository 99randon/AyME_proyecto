%%%%%%%%%%%%%%%%%%%%%%%%%%%%% Define Article %%%%%%%%%%%%%%%%%%%%%%%%%%%%%%%%%%
\documentclass{article}
%%%%%%%%%%%%%%%%%%%%%%%%%%%%%%%%%%%%%%%%%%%%%%%%%%%%%%%%%%%%%%%%%%%%%%%%%%%%%%%

%%%%%%%%%%%%%%%%%%%%%%%%%%%%% Using Packages %%%%%%%%%%%%%%%%%%%%%%%%%%%%%%%%%%
\usepackage{geometry}
\usepackage{graphicx}
\usepackage{amssymb}
\usepackage{amsmath}
\usepackage{amsthm}
\usepackage{empheq}
\usepackage{mdframed}
\usepackage{booktabs}
\usepackage{lipsum}
\usepackage{graphicx}
\usepackage{color}
\usepackage{psfrag}
\usepackage{pgfplots}
\usepackage{bm}
\usepackage{fancyhdr}
\usepackage{fontspec}
\usepackage[spanish]{babel}
\usepackage{datetime}
%%%%%%%%%%%%%%%%%%%%%%%%%%%%%%%%%%%%%%%%%%%%%%%%%%%%%%%%%%%%%%%%%%%%%%%%%%%%%%%

\setmainfont{calibri}[BoldFont = Calibri Bold]

%%%%%%%%%%%%%%%%%%%%%%%%%% Page Setting %%%%%%%%%%%%%%%%%%%%%%%%%%%%%%%%%%%%%%%
\geometry{a4paper}
\geometry{top=1.9cm, bottom=3.67cm, left=1.9cm, right=1.32cm}
\pagestyle{fancy}
\fancyhf{}
\lhead{\textbf{UNCuyo – Ing. Mecatrónica}\\Mendoza - Argentina}
\chead{\textbf{311 – AUTOMÁTICA Y MÁQUINAS ELÉCTRICAS}\\ 
\textbf{PROYECTO GLOBAL INTEGRADOR} Año: 2022}
\rhead{Alumnos: Mamani , Vignolo\\  \today}
\rfoot{Página \thepage{} de \pageref{LastPage}}
\renewcommand{\footrulewidth}{0.4pt}
\renewcommand{\headrulewidth}{0.4pt}
%%%%%%%%%%%%%%%%%%%%%%%%%%%%%%%%%%%%%%%%%%%%%%%%%%%%%%%%%%%%%%%%%%%%%%%%%%%%%%%

\begin{document}

\begin{titlepage}
    \centering
    \vspace*{3cm}
    \Huge\textbf{\textsc{Control de Accionamiento de CA con Motor Sincronico de Imanes Permanentes}}\\
    \vspace{1.5cm}
    \large{Subtitulo}\\
    \vspace{10cm}
    \large{Autors:}\\
    \large{\textbf{Alan Vignolo\\Brandon Mamani}}\\
    \vfill
    \the\year\\ % o puedes especificar una fecha específica en lugar de \today
\end{titlepage}

\part*{Introducción}

\lipsum

\newpage
\part*{Desarrollo}

\section*{Modelado, Análisis y Simulación dinámica del SISTEMA FÍSICO a Lazo Abierto}
\subsection*{Modelo matemático equivalente del subsistema mecánico completo}

Subsistema mecánico del motor:
\begin{equation}
    \label{eq:1.1}
    J_{m}\dot{\omega}_{m}(t) = T_{m}(t)-b_{m}\omega_{m}(t)-T_{d}(t) \\
\end{equation}
\begin{equation}
    \label{eq:1.2}
    \dot{\theta}_{m} = \omega_{m}
\end{equation}

Subsistema de tren de transmisión:
\begin{equation}
    \label{eq:2.1}
    \omega_{l}(t) = \frac{1}{r}\omega_{m}(t) \\
\end{equation}
\begin{equation}
    \label{eq:2.2}
    T_{q}(t) = rT_{d}(t) \\
\end{equation}

Subsistema de la carga mecánica:
\begin{equation}
    \label{eq:3.1}
    J_{l}\dot{\omega}_{l}(t) = T_{q}(t)-b_{l}\omega_{l}(t)-T_{l}(t) \\
\end{equation}
\begin{equation}
    \label{eq:3.2}
    \dot{\theta}_{l} = \omega_{l}\\
\end{equation}

Sistema mecánico equivalente completo:
\begin{equation}
    \label{eq:4.1}
    (J_{m}+\frac{J_{l}}{r^2})\dot{\omega}_{m}(t) = T_{m}(t)-(b_{m}+\frac{b_{l}}{r^2})\omega_{m}(t)-\frac{T_{d}(t)}{r} \\
\end{equation}

Con este modelo matemático equivalente referido al eje del motor tiene como 
ventaja que no presenta backlash, ademas nno hay que considerar el efecto de 
la elasticidad torsional de la transmisión.

\subsection*{Modelo dinámico del sistema físico completo}
\subsubsection*{Modelo global no lineal (NL)}

El modelo global no lineal considera tanto el sistema mecánico, previamente desarrollado
, como los subsistemas electromagnético y térmico. 
En primer lugar, nos enfocaremos en el subsistema electromagnético, teniendo en cuenta que 
se utiliza un motor síncrono de corriente alterna (CA) trifásico con imanes permanentes.

Ecuaciones de tensión en coordenadas abc:
\begin{equation}
    \begin{aligned}
        V_{a_{s}} &= R_{s}i_{a_{s}} + \frac{d\lambda_{a_{s}}}{dt}\\
        V_{b_{s}} &= R_{s}i_{b_{s}} + \frac{d\lambda_{b_{s}}}{dt}\\
        V_{c_{s}} &= R_{s}i_{c_{s}} + \frac{d\lambda_{c_{s}}}{dt}\\
    \end{aligned}
\end{equation}

Mediante la transformada de Park se obtiene:
\begin{equation}
    \begin{aligned}
        V_{q_{s}} &= R_{s}i_{q_{s}} + L_{q}\frac{d\lambda_{q_{s}}}{dt} + [\lambda_{m}^r + L_{d}i_{d_{s}}]\omega_{r}\\
        V_{d_{s}} &= R_{s}i_{d_{s}} + L_{d}\frac{d\lambda_{d_{s}}}{dt} - L_{q}i_{q_{s}}\omega_{r}\\
        V_{0_{s}} &= R_{s}i_{0_{s}} + L_{ls}\frac{d\lambda_{0_{s}}}{dt}\\
    \end{aligned}
\end{equation}

Estas ecuaciones representan...
Dada la conexión que presenta el motor, podemos suponer que la corriente $i_{0_{s}}$ es nula

El subsistema térmico:
\begin{equation}
    \frac{dT_{s}(t)}{dt} = \frac{\frac{3}{2}r_{s}({i_{a_{s}}^r(t)}^2+{i_{a_{s}}^r(t)}^2)-\frac{T_{s}(t)-T_{amb}(t)}{R_{ts-amb}}}{C_{ts}}
\end{equation}

El modelo global:
\begin{equation}
    \begin{cases}
        \dot{\theta}_{m}(t) = \omega_{m}(t)\\
        \dot{\omega}_{m}(t) = \frac{T_{m}}{T_{eq}} - \frac{b_{eq}}{T_{eq}}\omega_{m}(t) - \frac{T_{l}}{rT_{eq}}\\
        \dot{i}_{q_{s}}(t) = \frac{1}{L_{q}}[V_{q_{s}}^r - R_{s}i_{q_{s}}^r - P_{p}\omega_{m}(t)[L_{d}i_{d_{s}}^r+\lambda_{m}]]  \\
        \dot{i}_{d_{s}}(t) = \frac{1}{L_{d}}[V_{d_{s}}^r - R_{s}i_{d_{s}}^r + P_{p}\omega_{m}(t)L_{q}i_{q_{s}}^r]  \\
        \dot{T}_{s}(t) = \frac{\frac{3}{2}r_{s}({i_{a_{s}}^r(t)}^2+{i_{a_{s}}^r(t)}^2+2i_{0_{s}}^2(t))-\frac{T_{s}(t)-T_{amb}(t)}{R_{ts-amb}}}{C_{ts}}\\
    \end{cases}
\end{equation}

\subsubsection*{Modelo global linealizado con parámetros variables (LPV)}


\subsubsection*{Linealización por Realimentación NL}
\subsubsection*{Restricción o Ley de Control mínima}
\subsubsection*{Implementación}

\newpage
\begin{equation}
    \begin{bmatrix}
    a_{11} & a_{12} \\
    a_{21} & a_{22} \\
    \end{bmatrix} +
    \begin{bmatrix}
    \frac{d\lambda_{a_{s}}}{dt} & \frac{d\lambda_{a_{s}}}{dt}\\ 
    \frac{d\lambda_{a_{s}}}{dt} & \frac{d\lambda_{a_{s}}}{dt}\\ 
    \end{bmatrix} =
    \begin{bmatrix}
    c_{11} & c_{12} \\
    c_{21} & c_{22} \\
    \end{bmatrix}
\end{equation}
\begin{equation}
    \begin{cases}
        \begin{bmatrix}
        \frac{d\lambda_{a_{s}}}{dt} & \frac{d\lambda_{a_{s}}}{dt}\\ 
        \frac{d\lambda_{a_{s}}}{dt} & \frac{d\lambda_{a_{s}}}{dt}\\ 
        \end{bmatrix} \\
        2x - y = 1
    \end{cases}
\end{equation}

\newpage
\part*{Conclusiones}

\newpage
\part*{Referencias}

\label{LastPage}    
\end{document}