%%%%%%%%%%%%%%%%%%%%%%%%%%%%% Define Article %%%%%%%%%%%%%%%%%%%%%%%%%%%%%%%%%%
\documentclass{article}
%%%%%%%%%%%%%%%%%%%%%%%%%%%%%%%%%%%%%%%%%%%%%%%%%%%%%%%%%%%%%%%%%%%%%%%%%%%%%%%

%%%%%%%%%%%%%%%%%%%%%%%%%%%%% Using Packages %%%%%%%%%%%%%%%%%%%%%%%%%%%%%%%%%%
\usepackage{geometry}
\usepackage{graphicx}
\usepackage{amssymb}
\usepackage{amsmath}
\usepackage{amsthm}
\usepackage{empheq}
\usepackage{mdframed}
\usepackage{booktabs}
\usepackage{lipsum}
\usepackage{graphicx}
\usepackage{color}
\usepackage{psfrag}
\usepackage{pgfplots}
\usepackage{bm}
\usepackage{fancyhdr}
\usepackage{fontspec}
\usepackage[spanish]{babel}
\usepackage{datetime}
\usepackage{enumitem}
\usepackage{hyperref}
%%%%%%%%%%%%%%%%%%%%%%%%%%%%%%%%%%%%%%%%%%%%%%%%%%%%%%%%%%%%%%%%%%%%%%%%%%%%%%%

\setmainfont{calibri}[BoldFont = Calibri Bold]
\hypersetup{colorlinks=true, linkcolor=blue}

%%%%%%%%%%%%%%%%%%%%%%%%%% Page Setting %%%%%%%%%%%%%%%%%%%%%%%%%%%%%%%%%%%%%%%
\geometry{a4paper}
\geometry{top=1.9cm, bottom=3.67cm, left=1.9cm, right=1.32cm}
\pagestyle{fancy}
\fancyhf{}
\lhead{\textbf{UNCuyo – Ing. Mecatrónica}\\Mendoza - Argentina}
\chead{\textbf{311 – AUTOMÁTICA Y MÁQUINAS ELÉCTRICAS}\\ 
\textbf{PROYECTO GLOBAL INTEGRADOR}}
\rhead{Alumnos: Mamani , Vignolo\\  \today}
\rfoot{Página \thepage{} de \pageref{LastPage}}
\renewcommand{\footrulewidth}{0.4pt}
\renewcommand{\headrulewidth}{0.4pt}
%%%%%%%%%%%%%%%%%%%%%%%%%%%%%%%%%%%%%%%%%%%%%%%%%%%%%%%%%%%%%%%%%%%%%%%%%%%%%%%

\begin{document}

\begin{titlepage}
    \centering
    \vspace*{3cm}
    \Huge\textbf{\textsc{Control de Accionamiento de CA con Motor Sincronico de Imanes Permanentes}}\\
    \vspace{1.5cm}
    \large{Subtitulo}\\
    \vspace{10cm}
    \large{Autors:}\\
    \large{\textbf{Alan Vignolo\\Brandon Mamani}}\\
    \vfill
    \the\year\\ % o puedes especificar una fecha específica en lugar de \today
\end{titlepage}

\part*{Introducción}

\lipsum

\newpage
\part*{Desarrollo}

\section*{Modelado, Análisis y Simulación dinámica del SISTEMA FÍSICO a Lazo Abierto}
\subsection*{Modelo matemático equivalente del subsistema mecánico completo}

Subsistema mecánico del motor de CA trifásica síncrono, con el rotor referido al estator - sistema inercial de referencia:
\begin{equation}
    \label{eq:1.1}
    J_{m}\dot{\omega}_{m}(t) = T_{m}(t)-b_{m}\omega_{m}(t)-T_{d}(t) \\
\end{equation}
\begin{equation}
    \label{eq:1.2}
    \dot{\theta}_{m} = \omega_{m}
\end{equation}

Subsistema de tren de transmisión:
\begin{equation}
    \label{eq:2.1}
    \omega_{l}(t) = \frac{1}{r}\omega_{m}(t) \\
\end{equation}
\begin{equation}
    \label{eq:2.2}
    T_{q}(t) = rT_{d}(t) \\
\end{equation}

Subsistema de la carga mecánica:
\begin{equation}
    \label{eq:3.1}
    J_{l}\dot{\omega}_{l}(t) = T_{q}(t)-b_{l}\omega_{l}(t)-T_{l}(t) \\
\end{equation}
\begin{equation}
    \label{eq:3.2}
    \dot{\theta}_{l} = \omega_{l}\\
\end{equation}

Sistema mecánico equivalente completo:
\begin{equation}
    \label{eq:4.1}
    (J_{m}+\frac{J_{l}}{r^2})\dot{\omega}_{m}(t) = T_{m}(t)-(b_{m}+\frac{b_{l}}{r^2})\omega_{m}(t)-\frac{T_{l}(t)}{r} \\
\end{equation}

\begin{equation}
    J_{eq}\dot{\omega}_{m}(t) = T_{m}(t)-b_{eq}\omega_{m}(t)-\frac{T_{l}(t)}{r} \\
\end{equation}

\begin{equation}
    \dot{\omega}_{m}(t) = \frac{1}{J_{eq}}[T_{m}(t) - b_{eq}\omega_{m}(t) - \frac{T_{l}(t)}{r}]\\
\end{equation}

\begin{equation}
\begin{cases}
    \begin{bmatrix}
        \dot{\theta}_{m}(t)\\
        \dot{\omega}_{m}(t)\\
    \end{bmatrix}
    =
    \begin{bmatrix}
        0 & 1\\
        0 & -\frac{b_{eq}}{J_{eq}}\\
    \end{bmatrix}
    \begin{bmatrix}
        \theta_{m}(t)\\
        \omega_{m}(t)\\
    \end{bmatrix}
    +
    \begin{bmatrix}
        0 & 0\\
        \frac{1}{J_{eq}} & -\frac{1}{J_{eq}r}\\
    \end{bmatrix}
    \begin{bmatrix}
        T_{m}(t)\\
        T_{l}(t)\\
    \end{bmatrix}\\
    y(t) = 
    \begin{bmatrix} 1 & 0\\ \end{bmatrix} 
    \begin{bmatrix} \theta_{m}(t)\\ \omega_{m}(t)\\ \end{bmatrix}
\end{cases}
\end{equation}

Con este modelo matemático equivalente referido al eje del motor tiene como 
ventaja que no presenta backlash, ademas no hay que considerar el efecto de 
la elasticidad torsional de la transmisión.

\subsection*{Modelo dinámico del sistema físico completo}
\subsubsection*{Modelo global no lineal (NL)}

El modelo global no lineal considera tanto el sistema mecánico, previamente desarrollado
, como los subsistemas electromagnético y térmico.\\

En primer lugar, nos enfocaremos en el subsistema electromagnético, teniendo en cuenta que 
se utiliza un motor síncrono de corriente alterna (CA) trifásico con excitación de imanes 
permanentes. El estator esta conectado en estrella con bornes abc accesible y neutro no accesible.
Consideramos que la carga de cada fase sera equivalente de forma que la conexión estrella este equilibrada.

Ecuaciones de tensión en coordenadas abc:
\begin{equation}
    \begin{aligned}
        V_{as}(t) &= R_{s}(t)i_{as}(t) + \frac{d\lambda_{as}}{dt}\\
        V_{bs}(t) &= R_{s}(t)i_{bs}(t) + \frac{d\lambda_{bs}}{dt}\\
        V_{cs}(t) &= R_{s}(t)i_{cs}(t) + \frac{d\lambda_{cs}}{dt}\\
    \end{aligned}
\end{equation}

Mediante la transformación de Park, que consiste en premultiplicar por la matriz de Park se obtiene:
\begin{equation}
    \begin{aligned}
        V_{qs}(t) &= R_{s}(t)i_{qs}(t) + L_{q}\dot{i}_{qs}^r(t) + [\lambda_{m}^{\prime r} + L_{d}i_{ds}(t)]\omega_{r}(t)\\
        V_{ds}(t) &= R_{s}(t)i_{ds}(t) + L_{d}\dot{i}_{ds}^r(t)  - L_{q}i_{qs}(t)\omega_{r}(t)\\
        V_{0s}(t) &= R_{s}(t)i_{0s}(t) + L_{ls}\dot{i}_{0s}^r(t) \\
    \end{aligned}
\end{equation}

\begin{equation}\label{eq:mi_ecuacion}
    \begin{cases}
        \dot{i}_{qs}(t) = \frac{1}{L_{q}}[V_{qs}^r(t) - R_{s}(t)i_{qs}^r(t) - P_{p}\omega_{m}(t)[L_{d}i_{ds}^r(t)+\lambda_{m}^{\prime r}]]  \\
        \dot{i}_{ds}(t) = \frac{1}{L_{d}}[V_{ds}^r(t) - R_{s}(t)i_{ds}^r(t) + P_{p}\omega_{m}(t)L_{q}i_{qs}^r(t)]  \\
        \dot{i}_{0s}(t) = \frac{1}{L_{ls}}[V_{0s}^r(t) - R_{s}(t)i_{0s}^r(t)]  \\
    \end{cases}
\end{equation}

%Dudas:
Dada la conexión que presenta el motor, podemos suponer que la corriente $i_{0s}$ es nula

El subsistema térmico
Solo se consideran las pérdidas eléctricas resistivas causadas por el efecto Joule (calor),
 despreciando las pérdidas magnéticas en el núcleo y las transferencia de calor por conducción 
y convección natural. La potencia de pérdidas calóricas está dada por:\\
\begin{equation}
    P_{s_{perd}}(t) =  \frac{3}{2}R_{s}(t)(i_{qs}^r(t)^2+i_{ds}^r(t)^2+2.i_{0s}(t))\\
\end{equation}
\par El balance térmico en el estator:
\begin{equation}
    P_{s_{perd}}(t) = cs\dot{T}_{s}(t) + \frac{T_{s}(t)-T_{amb}(t)}{R_{ts-amb}}
\end{equation}

% 16
Torque electromagnético:
\begin{equation}\label{eq.torque_electromagnetico}
    T_{m}(t) = \frac{3}{2}P_{p}[\lambda_{m}^{\prime r}+i_{ds}^r(t)(L_{d}-L_{q})]i_{qs}^r(t)
\end{equation}

El modelo global:
\begin{equation}
    \begin{cases}
        \dot{\theta}_{m}(t) = \omega_{m}(t)\\
        \dot{\omega}_{m}(t) = \frac{1}{J_{eq}}[\frac{3}{2}P_{p}[\lambda_{m}^{\prime r}+i_{ds}^r(t)(L_{d}-L_{q})]i_{qs}^r(t) - b_{eq}\omega_{m}(t) - \frac{T_{l}(t)}{r}]\\
        \dot{i}_{qs}(t) = \frac{1}{L_{q}}[V_{qs}^r(t) - R_{s}(t)i_{qs}^r(t) - P_{p}\omega_{m}(t)[L_{d}i_{ds}^r(t)+\lambda_{m}^{\prime r}]]  \\
        \dot{i}_{ds}(t) = \frac{1}{L_{d}}[V_{ds}^r(t) - R_{s}(t)i_{ds}^r(t) + P_{p}\omega_{m}(t)L_{q}i_{qs}^r(t)]  \\
        \dot{T}_{s}(t) = \frac{1}{C_{ts}}[\frac{3}{2}R_{s}(t)({i_{qs}^r(t)}^2+{i_{ds}^r(t)}^2+2i_{0s}(t)^2)-\frac{T_{s}(t)-T_{amb}(t)}{R_{ts-amb}}]\\
    \end{cases}
\end{equation}


%-------------------------- 2B------------------------------------------%

\subsubsection*{Modelo global linealizado con parámetros variables (LPV)}

Para el caso general en el que ${i}_{ds}(t) \neq 0$... 

\begin{equation}
    \begin{cases}
        \dot{x}(t) = f(x(t),u(t));  \quad   x(t_{0}) = x_{0}\\
        y(t) = Cx(t)\\
    \end{cases}
\end{equation}

% texto

\begin{equation}
    \dot{x}(t) = 0 = f(x(t),u(t))
\end{equation}

%texto

\begin{equation}
    \begin{cases}
        x(t) = X_{0}(t) + \Delta x(t)\\
        u(t) = U_{0}(t) + \Delta u(t)\\
        y(t) = Y_{0}(t) + \Delta y(t)\\
    \end{cases}
\end{equation}

%texto

\begin{equation}
    \begin{cases}
        \dot{x}(t) = \dot{X}_{0}(t) + \Delta\dot{x}(t) = f(X_{0}(t) + \Delta x(t), U_{0}(t) + \Delta u(t))\\
        X_{0}(0) + \Delta x(0) = x_{0} \quad\rightarrow\quad X_{0} = x_{0}, \Delta x(0) = 0 \\
        Y_{0}(t) + \Delta y(t) = C(X_{0}(t) + \Delta x(t)) \quad\rightarrow\quad Y_{0}(t) = C X_{0}(t); \Delta y(t) = C \Delta x(t)\\
    \end{cases}
\end{equation}

%texto

\begin{equation}
    f(X_{0}(t) + \Delta x(t), U_{0}(t) + \Delta u (t)) \approx f(X_{0}(t), U_{0}(t)) + \frac{\partial f}{\partial x}\mid_{0}\Delta x(t) + \frac{\partial f}{\partial u}\mid_{0}\Delta u(t)
\end{equation}

Parte no lineal que representa el espacio de operación global NL:

\begin{equation}
    \dot{X}_{0}(t) = f(X_{0}(t), U_{0}(t)) \quad \approx 0 \;/\; \text{cte}; \quad X_{0}(0) = x_{0}
\end{equation}

Parte lineal dinámica que representa las pequeñas variaciones alrededor de los puntos de operación:

\begin{equation}
    \Delta\dot{x}(t) = \frac{\partial f}{\partial x}\mid_{0}\Delta x(t) + \frac{\partial f}{\partial u}\mid_{0}\Delta u(t); \quad \Delta x(0) = 0\\
\end{equation}
% Seguir corrigiendo ya que tengo inconsistencias con I0
Cuasi-estacionario
\begin{equation}
    \begin{cases}
        \dot{\theta}_{m0} = \omega_{m0} = cte\\
        \dot{\omega}_{m0} = \frac{1}{J_{eq}}[\frac{3}{2}P_{p}[\lambda_{m}^{\prime r}+i_{d_{s0}}^r(L_{d}-L_{q})]i_{q_{s0}}^r - b_{eq}\omega_{m0} - \frac{T_{l0}}{r}] = 0\\
        \dot{i}_{q_{s0}} = \frac{1}{L_{q}}[V_{q_{s0}}^r - R_{s}(t)i_{q_{s0}}^r - P_{p}\omega_{m0}[L_{d}i_{d_{s0}}^r+\lambda_{m}^{\prime r}]] = 0\\
        \dot{i}_{d_{s0}} = \frac{1}{L_{d}}[V_{d_{s0}}^r - R_{s}(t)i_{d_{s0}}^r + P_{p}\omega_{m0}L_{q}i_{q_{s0}}^r] = 0 \\
        \dot{i}_{0_{s0}} = \frac{1}{L_{ls}}[V_{0_{s0}} - R_{s}(t)i_{0_{s0}}] = 0\\
        \dot{T}_{s0} = \frac{1}{C_{ts}}[\frac{3}{2}R_{s}(t)({i_{q_{s0}}^r}^2+{i_{d_{s0}}^r}^2+2i_{0_{s0}}^2)-\frac{T_{s0}-T_{amb0}}{R_{ts-amb}}] = 0\\
    \end{cases}
\end{equation}

Parámetros Variables
\begin{equation}
    \begin{cases}
        \Delta\dot{\theta}_{m}(t) = \Delta\omega_{m}(t)\\
        \Delta\dot{\omega}_{m}(t) = \frac{1}{J_{eq}}[\frac{3}{2}P_{p}\{[\lambda_{m}^{\prime r} + i_{d_{s0}}^r(L_{d}-L_{q})]\Delta i_{qs}^r(t) + \Delta i_{ds}^r(t)(L_{d}-L_{q})]i_{q_{s0}}^r\}- b_{eq}\Delta\omega_{m}(t) - \frac{\Delta T_{l}(t)}{r}]\\
        \Delta\dot{i}_{qs}(t) = \frac{1}{L_{q}}[\Delta V_{qs}^r(t) - R_{s}(t)\Delta i_{qs}^r(t) - P_{p}\Delta\omega_{m}(t)[L_{d}i_{d_{s0}}^r+\lambda_{m}^{\prime r}] - P_{p}\omega_{m0}L_{d}\Delta i_{ds}^r(t)] \\
        \Delta\dot{i}_{ds}(t) = \frac{1}{L_{d}}\Delta [V_{ds}^r(t) - R_{s}(t)\Delta i_{ds}^r(t) + P_{p}\Delta\omega_{m}(t)L_{q}i_{q_{s0}}^r + P_{p}\omega_{m0}L_{q}\Delta i_{qs}^r(t)]  \\
        \Delta\dot{i}_{0s}(t) = \frac{1}{L_{ls}}[\Delta V_{0s}(t) - R_{s}(t)\Delta i_{0s}(t)] \\
        \Delta\dot{T}_{s}(t) = \frac{1}{C_{ts}}\{\frac{3}{2}R_{s}(t)[{2i_{qs}^r\Delta i_{ds}^r(t)} + 2i_{as}^r\Delta i_{as}^r(t)+4i_{0s}\Delta i_{0s}(t)]-\frac{\Delta T_{s}(t)-\Delta T_{amb}(t)}{R_{ts-amb}}\}\\
    \end{cases}
\end{equation}

\begin{multline}
    \begin{bmatrix}
        \Delta\dot{\theta}_{m}(t)\\
        \Delta\dot{\omega}_{m}(t)\\
        \Delta\dot{i}_{qs}(t)\\
        \Delta\dot{i}_{ds}(t)\\
        \Delta\dot{i}_{0s}(t)\\
        \Delta\dot{T}_{s}(t)\\
    \end{bmatrix} =
    \begin{bmatrix}
        0 & 1 & 0 & 0 & 0  & 0 \\
        0 & -b_{eq} & \frac{3}{2}P_{p}\frac{[\lambda_{m}^{\prime r} + i_{q_{s0}}^r(t)(L_{d}-L_{q})]}{J_{eq}} & \frac{3}{2}\frac{P_{p}(L_{d}-L_{q})i_{qs}^r}{J_{eq}} & 0 & 0\\
        0 & -\frac{P_{p}(\lambda_{m}^{\prime r} + L_{d}i_{d_{s0}}^r)}{L_{q}} & -\frac{R_{s}(t)}{L_{q}} & -\frac{L_{d}P_{p}\omega_{m0}}{L_{q}} & 0 & 0\\
        0 & \frac{P_{p}i_{q_{s0}}^r L_{q}}{L_{d}} & \frac{L_{q}P_{p}\omega_{m0}}{L_{d}} & -\frac{R_{s}(t)}{L_{d}} & 0 & 0\\
        0 & 0 & 0 & 0 & -\frac{R_{s}(t)}{L_{ls}} & 0\\
        0 & 0 & \frac{3R_{s}(t)}{C_{ts}}i_{q_{s0}}^r & \frac{3R_{s}(t)}{C_{ts}}i_{d_{s0}}^r & \frac{6R_{s}(t)}{C_{ts}}i_{0_{s0}} & -\frac{1}{C_{ts}R_{ts-amb}}\\
    \end{bmatrix}
    \begin{bmatrix}
        \Delta{\theta}_{m}(t)\\
        \Delta{\omega}_{m}(t)\\
        \Delta{i}_{qs}(t)\\
        \Delta{i}_{ds}(t)\\
        \Delta{T}_{s}(t)\\
    \end{bmatrix}\\  + 
    \begin{bmatrix}
        0 & 0 & 0 & 0 & 0\\
        -\frac{1}{rJ_{eq}} & 0 & 0 & 0 & 0\\
        0 & \frac{1}{L_{q}} & 0 & 0 & 0\\
        0 & 0 & \frac{1}{L_{d}} & 0 & 0\\
        0 & 0 & 0 & 0 & \frac{1}{C_{ts}R_{ts-amb}}\\
    \end{bmatrix}
    \begin{bmatrix}
        \Delta{T}_{l}(t)\\
        \Delta{V}_{qs}(t)\\
        \Delta{V}_{ds}(t)\\
        XXXX\\
        \Delta T_{amb}(t)\\
    \end{bmatrix} 
\end{multline}


%-------------------- 2C --------------------------%

\subsubsection*{Linealización por Realimentación NL}

Se busca un modelo simplificado lineal invariante LTI equivalente al modelo NL del sistema, 
para ello se propone un controlador de realimentación de estados que permita linealizar el 
modelo global NL obtenido anteriormente, por lo que se propone:

\begin{itemize}
    \item Aplicar la estrategia de “Control Vectorial con campo orientado” la cual 
    consiste en desacoplar los canales de flujo magnético y torque, lo cual provoca
    un forzamiento de $i_{ds}=0$ 
    \item Desacoplar el subsistema térmico, ya que se considera que hay variaciones despreciables 
    de Rs en el rango de temperaturas de trabajo.
\end{itemize}

\begin{enumerate}[label=\roman*.]
    \item Ecuaciones vectoriales/matriciales LTI de estado y de salida. 
    Matrices del modelo LTI equivalente.

    \begin{equation}
        \begin{cases}
            \dot{\theta}_{m}(t) = \omega_{m}(t)\\
            \dot{\omega}_{m}(t) = \frac{1}{J_{eq}}[\frac{3}{2}P_{p}\lambda_{m}^{\prime r}i_{qs}^r(t) - b_{eq}\omega_{m}(t) - \frac{T_{l}(t)}{r}]\\
            \dot{i}_{qs}(t) = \frac{1}{L_{q}}[V_{qs}^r(t) - R_{s}i_{qs}^r(t) - P_{p}\omega_{m}(t)\lambda_{m}^{\prime r}]  \\
        \end{cases}
    \end{equation}

    \begin{equation}
        \begin{cases}
            \begin{bmatrix}
                \dot{\theta}_{m}(t)\\
                \dot{\omega}_{m}(t)\\
                \dot{i}_{qs}(t)\\
            \end{bmatrix} =
            \begin{bmatrix}
                0 & 1 & 0\\
                0 & -\frac{b_{eq}}{J_{eq}} & \frac{3}{2}\frac{P_{p}\lambda_{m}^{\prime r}}{J_{eq}}\\
                0 & -\frac{P_{p}\lambda_{m}^{\prime r}}{L_{q}} & -\frac{R_{s}}{L_{q}}\\
            \end{bmatrix}
            \begin{bmatrix}
                {\theta}_{m}(t)\\
                {\omega}_{m}(t)\\
                {i}_{qs}(t)\\
            \end{bmatrix} +
            \begin{bmatrix}
                0\\
                0\\
                \frac{1}{L_{eq}}\\
            \end{bmatrix} V_{qs}^r(t) +
            \begin{bmatrix}
                0\\
                -\frac{1}{rJ_{eq}}\\
            \end{bmatrix} T_{l}(t)\\
            y(t) =
            \begin{bmatrix}
                1 & 0 & 0\\
            \end{bmatrix}
            \begin{bmatrix}
                {\theta}_{m}(t)\\
                {\omega}_{m}(t)\\
                {i}_{qs}(t)\\
            \end{bmatrix}
        \end{cases}
    \end{equation}
        
    \item Segundo ítem
    
    
    
    \item Restricción o Ley de Control mínima

    \begin{equation}
        V_{ds}^r(t) = -L_{q}i{qs}^r(t)\omega_{m}(t)P_{p}
    \end{equation}

    Se aplica la inversa de Park

    \begin{equation}
        \begin{cases}
            V_{as}(t) = cos(\theta_{r}(t))V_{qs}^r(t) + sin(\theta_{r}(t))V_{ds}^r(t) + V_{0s}^r(t)\\
            V_{bs}(t) = cos(\theta_{r}(t) - \frac{2\pi}{3})V_{qs}^r(t) + sin(\theta_{r}(t) - \frac{2\pi}{3})V_{ds}^r(t) + V_{0s}^r(t)\\
            V_{cs}(t) = cos(\theta_{r}(t) + \frac{2\pi}{3})V_{qs}^r(t) + sin(\theta_{r}(t) + \frac{2\pi}{3})V_{ds}^r(t) + V_{0s}^r(t)\\
        \end{cases}
    \end{equation}

    Sustituyendo

    \begin{equation}
        \begin{cases}
            V_{as}(t) = cos(\theta_{r}(t))V_{qs}^r(t) - sin(\theta_{r}(t))L_{q}i{qs}^r(t)\omega_{m}(t)P_{p}\\
            V_{bs}(t) = cos(\theta_{r}(t) - \frac{2\pi}{3})V_{qs}^r(t) - sin(\theta_{r}(t) - \frac{2\pi}{3})L_{q}i{qs}^r(t)\omega_{m}(t)P_{p}\\
            V_{cs}(t) = cos(\theta_{r}(t) + \frac{2\pi}{3})V_{qs}^r(t) + sin(\theta_{r}(t) - \frac{2\pi}{3})L_{q}i{qs}^r(t)\omega_{m}(t)P_{p}\\
        \end{cases}
    \end{equation}

    \item Dinámica Residual
    
    \begin{equation}
        \begin{aligned}
        \frac{di_{ds}^r(t)}{dt} = \frac{1}{L_{d}}[-R_{s}(t)i_{ds}^r(t)] \\
        \frac{di_{ds}^r(t)}{dt} + \frac{R_{s}(t)}{L_{d}}i_{ds}^r(t) = 0 \\
        \end{aligned}
    \end{equation} 

    \begin{equation}
        i_{ds}^r(t) = i_{ds}^r(0)e^{-\frac{R_{s}(t)}{L_{d}}t}
    \end{equation}

    \begin{equation}
        V_{qs}(t) = L_{q}\frac{di_{qs}^r(t)}{dt} + R_{s}i_{qs}^r(t) + P_{p}\omega_{m}(t)\lambda_{m}^{\prime r} + \mathbf{L_{d}i_{ds}^r(t)P_{p}\omega_{m}(t)}
    \end{equation}

    \begin{equation}
        \begin{cases}
            \dot{\theta}_{m}(t) = \omega_{m}(t)\\
            \dot{\omega}_{m}(t) = -\frac{1}{J_{eq}}[\frac{3}{2}P_{p}\lambda_{m}^{\prime r} i_{qs}^r(t) - b_{eq}\omega_{m}(t)-\frac{T_{l}(t)}{r}]\\
            \dot{i}_{qs}^r(t) = \frac{1}{L_{q}}[V_{qs}^r(t) - R_{s}i_{qs}^r(t) - P_{p}\omega_{m}(t)\lambda_{m}^{\prime r}]\\
            \dot{T}_{s}(t) = \frac{1}{C_[ts]}\{\frac{3}{2}R_{s}(t)[{i_{qs}^r}^2(t) + {i_{ds}^r}^2(t)] - \frac{1}{R_{ts-amb}}[T_{s}(t) - T_{amb}(t)]\}\\
            \dot{i}_{ds}^r(t) = -\frac{R_{s}(t)}{L_{d}}i_{ds}^r(t) \\
        \end{cases}    
    \end{equation}

\end{enumerate}

%-------------------- 2D --------------------------%

\subsubsection*{Comparación modelo dinámico LTI equivalente aumentado vs modelo dinámico global LPV}

El modelo dinámico global LPV para el caso general donde $i_{ds}^r \neq 0$
es una mejor representación del sistema real al tener mejor representadas sus 
no linealidades.
En cambio el modelo LTI, donde $i_{ds}^r = 0$, tiene la ventaja de un modelo 
con un mayor grado de simplicidad. Sin embargo, esta simplificación  causa que 
el espacio de puntos de operación se reduzca.
Analizaremos el comportamiento del sistema frente a cambios de $i_{ds}^r$, 
considerando el estado estacionario. 

\begin{itemize}
    
    \item Respecto al par electromagnético:

    \begin{equation}
        T_{m}(t) = \frac{3}{2}P_{p}[\lambda_{m}^{\prime r} + (L_{d}-L_{q})i_{ds}^r(t)]
    \end{equation}

    Para motores de polos salientes $L_{d} > L_{q}$, entonces cuando $i_{ds}^r(t)$
    toma valores positivos el campo magnético se refuerza lo que aumenta el torque del
    motor. Si la corriente directa toma valores negativos el campo magnético se debilita 
    y disminuye el torque del motor. 

    \item Respecto al subsistema eléctrico:

    % Voltaje con minúsculas
    \begin{equation}
        \dot{i}_{ds}^r = \frac{1}{L_{d}}[v_{ds}(t) - R_{s}(t)i_{ds}^r(t) + L_{q}i_{qs}^r(t)P_{p}\omega_[m](t)] = 0
    \end{equation}
    \begin{equation}
        \omega_{m}(t) = \frac{-v_{ds}(t) + R_{s}(t)i_{ds}^r(t)}{L_{q}}
    \end{equation}

    En este caso, la velocidad del motor disminuye cuando la corriente $i_{ds}^r$ 
    aumenta. Por lo que podemos concluir que el torque reaccionara inversamente 
    a la velocidad.

\end{itemize}

En el caso que $i_{ds}^r(t) = 0$, el flujo concatenado solamente esta afectado por 
los imanes permanentes.

%-------------------- 2E --------------------------%

\subsubsection*{Funciones de transferencia para el modelo LTI}

Las funciones de transferencia nos permiten relacionar las salidas con las entradas del sistema.
Como nuestro sistema dispone de dos entradas obtendremos dos funciones de transferencia.

Para comenzar, es necesario aplicar la transformada de Laplace a las ecuaciones del modelo LTI:

\begin{equation}
        L[f{t}] = F(s)
\end{equation}

La transformada de Laplace posee la siguiente propiedad

\begin{equation}
        L[\dot{f}(t)] = s F(s) - f(0)
\end{equation}

Al aplicar la transformada y recordando que las condiciones iniciales son nulas, el 
modelo queda

\begin{equation}
    \begin{cases}
        s\Theta_{m}(s) = \Omega_{m}(s)\\
        s\Omega_{m}(s) = \frac{1}{J_{eq}}[\frac{3}{2} P_{p} \lambda_{m}^{\prime r} I_{qs}^r(s) - b_{eq}\Omega(s) - \frac{T_{l}(s)}{r}]\\
        s I_{qs}^r(s) = \frac{1}{L_{q}}[V_{qs}^r(s) - R_{s}I_{qs}^r(s) - P_{p}\Omega_{m}(s)\lambda_{m}^{\prime r}]\\
    \end{cases}
\end{equation}

Para obtener las funciones de transferencia del modelo, se despeja $I_{qs}^r(s)$

\begin{equation}
    I_{qs}^r(s) = \frac{V_{qs}^r(s) - P_{p}\Omega_{m}(s)\lambda_{m}^{\prime r}}{s L_{q} + R_{s}}
\end{equation}
 
Remplazando en las ecuaciones de estado 

\begin{equation}
    \Theta_{m}(s) = \frac{\frac{3}{2}.P_{p}.\lambda_{m}^{\prime r}.V_{qs}(s) - \frac{1}{r}.(s.L_{q}+R_{s}).T_{l}(s)}{s^3.J_{eq}.L_{q} + s^2.(b_{eq}.L_{q} +  R_{s}.J_{eq}) + s.[b_{eq}.R_{s}+\frac{3}{2}.(P_{p}.\lambda_{m}^{\prime r})^2]}
\end{equation}

De esta expresion se obtienen las funciones de transferencia respecto de las entradas tension $V_{qs}(s)$ y torque $T_{l}(s)$

\begin{equation}
    G_{1}(s) = \frac{\Theta_{m}(s)}{V_{qs}(s)} = \frac{\frac{3}{2}.P_{p}.\lambda_{m}^{\prime r}}{s^3.J_{eq}.L_{q} + s^2.(b_{eq}.L_{q} +  R_{s}.J_{eq}) + s.[b_{eq}.R_{s}+\frac{3}{2}.(P_{p}.\lambda_{m}^{\prime r})^2]}
\end{equation}

\begin{equation}
    G_{2}(s) = \frac{\Theta_{m}(s)}{T_{l}(s)} = \frac{- \frac{1}{r}.(s.L_{q}+R_{s})}{s^3.J_{eq}.L_{q} + s^2.(b_{eq}.L_{q} +  R_{s}.J_{eq}) + s.[b_{eq}.R_{s}+\frac{3}{2}.(P_{p}.\lambda_{m}^{\prime r})^2]}
\end{equation}


%-------------------- 5.2 --------------------------%


\section*{Diseño, análisis y simulación con controlador de movimiento en cascada con modulador de torque equivalente (Control Vectorial)}

El control vectorial es una técnica avanzada utilizada para controlar motores de corriente alterna y máquinas 
síncronas. Permite controlar de manera independiente la magnitud y la fase de la corriente de alimentación de la 
máquina para lograr un control preciso del torque y la velocidad. Una implementación común del control vectorial 
es el controlador de movimiento en cascada con modulador de torque equivalente, utilizado en aplicaciones de alto 
rendimiento como sistemas de tracción eléctrica y accionamiento de maquinaria industrial.

En el controlador de movimiento en cascada con modulador de torque equivalente, se utiliza un controlador de 
velocidad externo para generar una señal de referencia de velocidad que se compara con la velocidad real de la 
máquina. La diferencia entre estas señales se utiliza para generar una señal de referencia de torque, que controla 
la corriente de alimentación. El modulador de torque equivalente convierte la señal de referencia de torque en una
señal de corriente de alimentación mediante la técnica de modulación de ancho de pulso (PWM), controlando así la 
magnitud y la fase de la corriente de alimentación.

%-------------------- 1 --------------------------%

\subsection*{Modulador de torque equivalente (Controlador interno vectorial de corriente/torque)}

En el próximo paso de nuestro sistema, se llevará a cabo la implementación de un modulador de torque equivalente, 
conocido también como controlador interno vectorial de corriente/torque. Este controlador nos permitirá, de manera
similar a una máquina de corriente continua, controlar el sistema utilizando consignas de torque como entradas, 
las cuales serán posteriormente transformadas en consignas de tensión. Para lograrlo, procederemos a implementar 
el controlador completo con su correspondiente diagrama de bloques, utilizando el modelo NL completo y los valores
de parámetros adecuados, siguiendo los lineamientos que se detallarán a continuación.

%-------------------- 1.a --------------------------%

\subsubsection*{Desacople de las realimentaciones de estado hacia la entrada}

Considerando que el modulador de tensión es lo suficientemente rápido y preciso, podemos asumir que su ganancia 
es unitaria. Por lo tanto, podemos considerar que en la entrada del modulador de tensión se tiene una consigna 
de tensión a seguir, y esta misma tensión se aplicará a la salida del modulador. 
En otras palabras, en la salida del modulador se obtiene la tensión como variable física. Con base en lo 
mencionado, podemos establecer que:

\begin{equation}
    v_{abc}(t) \approx v^*_{abc}(t)
\end{equation}

Utilizando la transformada de Park, podemos definir una consigna de tensión en coordenadas virtuales 
$v^{r*}_{qd0s}(t)$ que nos permita obtener una tensión $v^{r}_{qd0s}(t)$ que compense los efectos de 
retroalimentación y permita el desacople de las variables de estado.

Recordando las \hyperref[eq:mi_ecuacion]{Ecuaciones \ref*{eq:mi_ecuacion}}, se observa que los términos 
del lado derecho de la igualdad, a excepción de las tensiones de fase virtuales, representan las 
realimentaciones físicas del sistema. Por lo tanto, podemos definir las compensaciones que deben 
realizarse en el controlador para cancelar los efectos de la retroalimentación, las cuales son:

\begin{equation}\label{eq:compensaciones}
    \begin{cases}
        v^r_{qs}(t) = v^{r*}_{qs}(t) + R_{s}.i_{qs}^r(t) + \omega_m(t).P_p.[\lambda^{\prime r}_m + L_d.i_{qs}^r(t)] \\
        v^r_{ds}(t) = v^{r*}_{ds}(t) + R_{s}.i_{ds}^r(t) - \omega_m(t).P_p.L_q.i_{qs}^r(t) \\
        v^r_{0s}(t) = v^{r*}_{0s}(t) + R_{s}.i_{0s}^r(t) \\
    \end{cases}
\end{equation}

Realizando estas compensaciones, tenemos acceso directo a manipular el torque electromagnético, 
sin los efectos de las realimentaciones físicas, ni las caídas de tensión en los bobinados.

Reemplazando las \hyperref[eq:compensaciones]{Ecuaciones \ref*{eq:compensaciones}} en las
\hyperref[eq:mi_ecuacion]{Ecuaciones \ref*{eq:mi_ecuacion}}, se obtienen las siguientes expresiones:

\begin{equation}\label{eq:proporcionlaidad_corrientes_tensiones_de_entrada}
    \begin{cases}
        v^{r*}_{qs}(t) = L_q. \dot{i}_{qs}^r(t) \\
        v^{r*}_{ds}(t) = L_d. \dot{i}_{ds}^r(t) \\
        v^{r*}_{0s}(t) = L_{ls}. \dot{i}_{0s}^r(t) \\
    \end{cases}
\end{equation}

% Aca muestra una imagen del modelo el modelo pag. 35 %

% En la siguiente página se muestra el modelo de bloques que considera el desacople de las 
% realimentaciones físicas, teniendo en cuenta la variación de la temperatura del sistema $(Ts(t))$. 
% En este modelo, se tiene en cuenta que la resistencia de los devanados del estator varía con la 
% temperatura. Se ha decidido mantener esta variación en la resistencia, ya que fue considerada en 
% el modelo no lineal con la ley de control mínima. Aunque se reconoce que la variación de la 
% resistencia con la temperatura es pequeña, mantener esta variación permite lograr acciones de control 
% más efectivas, replicando de manera más precisa el comportamiento real del sistema físico. Además, 
% el costo computacional de esta medida es mínimo.

%-------------------- 1.b --------------------------%

\subsubsection*{Diseño de lazos de control de corrientes}

La consigna de tensión es función de la corriente del sistema, por lo que se puede controlar usando 
una consigna de corriente proporcional. Con el error de corriente entre la corriente de consigna 
$i_{qd0s}^{r*}(t)$y la corriente real del sistema se modela $v_{qd0s}^{r*}(t)$ con una ley de 
control proporcional.. El modelo resultante es el siguiente:

\begin{equation}
    \begin{cases}
        L_q. \dot{i}_{qs}^r(t) = v^{r*}_{qs}(t) = [i_{qs}^{r*}(t) - i_{qs}^r(t)].R_q^\prime \\
        L_d. \dot{i}_{ds}^r(t) = v^{r*}_{ds}(t) = [i_{ds}^{r*}(t) - i_{ds}^r(t)].R_d^\prime \\
        L_{ls}. \dot{i}_{0s}^r(t) = v^{r*}_{0s}(t) = [i_{0s}^{r*}(t) - i_{0s}^r(t)].R_{0}^\prime \\
    \end{cases}
\end{equation}

Las variables $R^\prime$ representan las ganancias del control proporcional. Se necesita obtener 
el valor de óptimo de esta ganancia, por lo cual se analiza la función de transferencia del modulador.

Al aplicar la transformada de Laplace obtenemos:

\begin{equation}
    \begin{cases}
        L_q. s.I_{qs}^r(s) = [I_{qs}^{r*}(s) - I_{qs}^r(s)].R_q^\prime \\
        L_d. s.I_{ds}^r(s) = [I_{ds}^{r*}(s) - I_{ds}^r(s)].R_d^\prime \\
        L_{ls}. s.I_{0s}^r(s) = [I_{0s}^{r*}(s) - I_{0s}^r(s)].R_{0}^\prime \\
    \end{cases}
\end{equation}

Las funciones de transferencia son:

\begin{align*}
    G_{qs}(s) &= \frac{I_{qs}^r(s)}{I_{qs}^{r*}(s)} = \frac{1}{\frac{L_q}{R_q^\prime}.s + 1}  \\
    G_{ds}(s) &= \frac{I_{ds}^r(s)}{I_{ds}^{r*}(s)} = \frac{1}{\frac{L_d}{R_d^\prime}.s + 1}  \\
    G_{0s}(s) &= \frac{I_{0s}^r(s)}{I_{0s}^{r*}(s)} = \frac{1}{\frac{L_{ls}}{R_{0}^\prime}.s + 1}
\end{align*}

Se puede observar que las funciones de transferencia son de primer orden y no poseen ceros. Dado que
los parámetros son positivos también podemos afirmar que las funciones de transferencia son estables.

Estas funciones de transferencia tienen la forma típica de un filtro pasa bajos, donde el valor que
multiplica a la variable $s$ es la constante del tiempo del sistema \tau que a la vez es el inverso
la frecuencia de corte del filtro. Si \tau toma un valor muy pequeño, dispondremos de una ancho de 
banda muy grande, lo cual proporciona una respuesta rápida.

El polo del filtro pasa bajos es $p = -\frac{1}{\tau}$ y dado que para todos los lazos $p = -5000 rad/s$
podemos calcular los valores de las ganancias:

\begin{equation}
    \frac{L}{R^\prime} = -\frac{1}{p} \to R^\prime = -L.p
\end{equation}

Resolviendo para cada rama:

\begin{align*}
    R_q^\prime &= -L_q.p = 29\varOmega  \\
    R_d^\prime &= -L_d.p = 33\varOmega \\
    R_0^\prime &= -L_{ls}.p = 4\varOmega
\end{align*}

Con este lazo de corriente logramos que el error entre la consigna de corriente y la corriente real
converja mas rápidamente a cero, de forma que responda de mejor manera a las perturbaciones.

% Imágenes %

% Estaría bueno responder la diferencia entre la consigna 5.1.2.c.vi %

%-------------------- 1.c --------------------------%

\subsubsection*{Incorporación de consigna de torque}

Volvemos a usar nuevamente el método de la sección anterior en el que realizamos una realimentación
para controlar el sistema mediante consignas de torque $T_m^*(t)$. La consigna sigue el siguiente modelo:

\begin{equation}
    T_m^*(t) = T_m^{r\prime}(t) + b_{eq}.\omega_m(t)
\end{equation}

Este modelo contempla la realimentación física debida a la fricción, que genera perdidas en el torque.

Sabiendo que el torque y la corriente están relacionados por la ecuación del
\hyperref[eq.torque_electromagnetico]{Torque electromagnético (\ref*{eq.torque_electromagnetico})}
expresamos las consignas de troque $T_m^{*\prime}$ en función de las consignas de corriente $i_{qs}^{r*}$:

\begin{equation}
    T_m^{*}(t) = \frac{3}{2}.P_p.[\lambda_m^{\prime r} + (L_d - L_q)i_{ds}^{r}(t)].i_{qs}^{r*}(t)
\end{equation}

Despejando de estas ultimas dos ecuaciones, obtenemos:

\begin{equation}
    i_{qs}^{r*}(t) = \frac{T_m^{*\prime}(t) + b_{eq}.\omega_m(t)}{\frac{2}{3}.P_p.[\lambda_m^{\prime r} + (L_d - L_q)i_{ds}^{r}(t)]}
\end{equation}

% IMAGEN DEL MODULADOR DE TORQUE %

% DUDAS YA QUE NO TRABAJAMOS CON IDS %

De esta ecuación podemos determinar el efecto que tendrá la corriente $i_{ds}^{r*}$ sobre el flujo magnético:

\begin{itemize}
    \item Si $i_{ds}^{r*} > 0 A$ entonces se produce un reforzamiento de campo.
    \item Si $i_{ds}^{r*} = 0 A$ entonces se produce un desacople entre la rama correspondiente a 
    la cuadratura que anula los efectos de reforzamiento y debilitamiento.
    \item Si $i_{ds}^{r*} < 0 A$ entonces se produce un debilitamiento de campo.
\end{itemize}

\subsection*{Controlador externo de movimiento (posición/velocidad)}

El controlador externo de movimiento se agrega con el fin de mejorar la dinámica del sistema y corregir los
errores de estado estacionario producidos por cargas perturbadoras. En la rama derivativa tomaremos el 
error entre la consigna de velocidad y la velocidad real. El error de posición se obtiene integrando el 
error de velocidad lo que indica que no son variables independientes, este va en la rama proporcional.
En la rama integral se tendrá la integral del erro de posición. 

% Filtro pasa alto %
Con esta configuración del controlador PID solo tendremos como entrada el error de velocidades, no tendremos
que introducir acciones derivativas lo que nos permitirá evitar la amplificación del ruido y obtener un 
controlador mas estable. Los bloques integrales actúan como filtros pasa bajos lo que nos ayuda a eliminar
el ruido de los errores de posición y velocidad dada su naturaleza de alta frecuencia.

Se diseña el controlador utilizando el método de sintonía serie con $n = 2,5$, $\omega_{pos} = 800 rad/s$, y
considerando los valores nominales de $j_l$, $b_l$. 

El controlador de movimiento se muestra en la siguiente figura: 

% IMAGEN DEL PID %

La salida del controlador sera el torque consigna que ingresara al modulador de torque. Modelando en el 
dominio de Laplace se tiene:

\begin{equation}\label{eq.controlador_error}
    T_m^{*\prime}(s) = e_\omega(s).b_a + e_\theta(s).K_{sa} + e_\theta(s).K_{sia}.\frac{1}{s}
\end{equation}

Donde:

\begin{align*}
    e_\theta(s) &= \Theta_m^*(s) - \Theta_m(s) \\
    e_\omega(s) &= e_\theta(s).s
\end{align*}

La relación entre el torque y la variación de velocidad del motor en el modelo del subsistema mecánico, 
teniendo en cuanta el desacople de fricción realizado anteriormente, es:

\begin{equation}
    J_{eq}.\dot{\omega}_(t) = T_m^*(t) - \frac{T_l(t)}{r}
\end{equation}

Aplicando la transformada de Laplace:

\begin{equation}
    J_{eq}.s^2.\Theta_m(s) = T_m^*(s) - \frac{T_l(s)}{r}
\end{equation}

Reemplazando esta ecuación en la
\hyperref[eq.controlador_error]{ecuación del controlador (\ref*{eq.controlador_error})}:

\begin{equation}
    J_{eq}.s^3.\Theta_m(s) = [s^2.b_a + s.K_{sa} + K_{sia}].[\Theta_m^*(s) - \Theta_m(s)] - s.\frac{T_l(s)}{r}
\end{equation}

Despejando, la posición del retor queda expresada como:

\begin{equation}
    \Theta_m(s) = \frac{s^2.b_a + s.K_{sa} + K_{sia}}{s^3.J_{eq} + s^2.b_a + s.K_{sa} + K_{sia}}.\Theta_m^*(s) - \frac{s}{s^3.J_{eq} + s^2.b_a + s.K_{sa} + K_{sia}}.\frac{T_l(s)}{r}
\end{equation}

A partir de esta ecuación podemos obtener las funciones de transferencia del controlador:

\begin{align}
    G_1(s) &= \frac{\Theta_m(s)}{\Theta_m^*(s)} = \frac{s^2.b_a + s.K_{sa} + K_{sia}}{s^3.J_{eq} + s^2.b_a + s.K_{sa} + K_{sia}} \\
    G_2(s) &= \frac{\Theta_m(s)}{\frac{T_l(s)}{r}} = -\frac{s}{s^3.J_{eq} + s^2.b_a + s.K_{sa} + K_{sia}}
\end{align}

En régimen estacionario para una entrada escalón unitario podemos observar:

\begin{itemize}
    \item $K_{sai} \neq 0 \to G_1(s) = 1 $ y $G_2(s) = 0$
    \item $K_{sai} = 0 \to G_2(s) = 1 $ y $G_2(s) = \frac{1}{K_{sa}} $
\end{itemize}

Estos resultados muestran que que la función de transferencia
correspondiente a la entrada de perturbación tejen un cero en el origen por lo que el error de estado 
estacionario es nulo. En caso de que la acción integral sea nula si tendrá un error dado por $1/K_{sa}$

En cuanto a la función de transferencia correspondiente a la entrada de referencia tiene ganancia
unitaria a baja frecuencia por lo que el error de estado estacionario, según el teorema del valor final, 
es nulo. 

Podemos concluir que este  controlador no tiene error de régimen permanente ante entradas del tipo escalón.

























































































































































































































































































\newpage
\part*{Conclusiones}

\newpage
\part*{Referencias}
\lipsum{}
\label{LastPage}    
\end{document}